\begin{savequote}[72mm]
  Machine consciousness refers to attempts by those who design and
  analyse informational machines to apply their methods to various
  ways of understanding consciousness and to examine the possible role
  of consciousness in informational machines.

  \qauthor{Igor Aleksander}
\end{savequote}

\chapter{Redes Neuronales}
\label{cha:redes}

En este capítulo estudiaremos los conceptos principales de varias
redes neuronales diviéndolas en dos tipos: Redes neuronales
supervisadas y redes neuronales no supervisadas. Esta distinción está
dada por la forma en que se efectua el aprendizaje dentro de las redes neuronales
\section{Redes Neuronales Supervisadas}



\subsection{Perceptrón simple}



\subsection{Adalina}


\subsection{Perceptrón multicapa}


\subsection{Back Propagation}


\section{Redes Neuronales no Supervisadas}


\subsection{Mapas autoorganizados}


\subsubsection{Cuantificación optima de vectores}



\subsubsection{Modelo de neurona de Kohonen}


\subsubsection{Modelos de aprendizaje en mapas autoorganizados}



\subsection{Redes de Hopfield}



\subsection{Funciones de base radial}



\subsection{Aprendizaje vectorial cuantificado}
