\begin{prefacio}

  El presente documento es escrito con la intención de proveer un
  enfoque en la implementación de redes neuronales a través de
  software, esto debido a que la implementación en hardware no siempre
  es posible y suele estar fuera del alcance de los que buscan entrar
  en el mundo de las redes neuronales.

  El enfoque de la implementación es a través de un lenguaje funcional
  para definir la mayoría de las operaciones entre redes como
  funciones, esto para tratar de mantener un paralelismo entre la
  modelación matemática de la red neuronal y la implementación en
  código.

  El lenguaje elegido para realizar todas estas operaciones es
  Clojure, el cuál a mi parecer, las siguientes razones lo hacen un
  lenguaje adecuado para trabajar.

  \begin{itemize}
  \item \textbf{Filosofía} Clojure hace algunas muy buenas decisiones de
    diseño. Por ejemplo estructuras de datos inmutables, así como la
    tendencia a implementar funciones puras (i.e. no tienen efectos
    secundarios).
  \item \textbf{Concurrencia} Clojure es realmente bueno con problemas de
    concurrencia, y varias bibliotecas lo hacen mucho mejor.
  \item \textbf{Interoperabilidad} Clojure puede hacer uso de bibliotecas de
    Java existentes.
  \item \textbf{Desarrollo} La creación de código y el prototipado rápido
    permiten una mejor experiencia al programar.
  \item \textbf{Macros y DSL} Permite extender el lenguaje de forma muy
    sencilla a través de macros y con ello crear lenguajes de dominio
    específico (DSL - Domain Specific Languages)
  \end{itemize}

  Una vez elegido el lenguaje y la explicación del porqué ha sido
  elegido, explicaremos brevemente los tipos de redes neuronales que
  vamos a construir. Generalmente las redes neuronales se diferencian
  por el tipo de aprendizaje que usan, siendo el aprendizaje
  supervisado y el no supervisado los más comunes. Existen también
  soluciones híbridas las cuales mencionaremos brevemente en los
  capítulos dedicados a las redes neuronales. En el primer capítulo se
  hará una exposición un poco más detallada de porqué Clojure es una
  buena opción.

  En el segundo capítulo hablaremos de las siguientes redes:

  \begin{itemize}
  \item \textsl{Perceptrón Simple}
  \item \textsl{Adalina}
  \item \textsl{Perceptrón multicapa y backpropagation}
  \item \textsl{Mapas autoorganizados (Redes de Kohonen)}
  \item \textsl{Redes de Hopfield}
  \end{itemize}

  Así como su base matemática y algunas de las aplicaciones que
  tienen.

  En el tercer capítulo haremos una breve introducción a las distintas
  características de Clojure que serán fundamentales conocer al
  momento de implementar las redes neuronales mencionadas en el
  capítulo 2.

  En el cuarto capítulo explicaremos como realizar la implementación
  de las distintas redes neuronales. Continuando con el quinto
  capítulo en el cuál veremos una aplicación real de las redes
  neuronales y la comparación con otras técnicas.

  Y finalizamos el documento dando conclusiones sobre los temas
  desarrollados en el libro. Espero que el presente documento sea de
  su agrado y sirva como apoyo para la comprensión de las redes
  neuronales artificiales.


\end{prefacio}